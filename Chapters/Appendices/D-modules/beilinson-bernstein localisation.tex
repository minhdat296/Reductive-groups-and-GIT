\section{Beilinson-Bernstein Localisation}
    \subsection{Some supplementary Lie theory}
        \subsubsection{Lie algebras and their universal enveloping algebras}
            \begin{definition}[Lie algebras] \label{def: lie_algebras}
                Let $k$ be a ring (not necessarily commutative) and let $(\calV, \tensor, 1, \tau)$ be a symmetric monoidal $k$-linear category with braiding isomorphisms:
                    $$\tau_{x, y}: x \tensor y \cong y \tensor x$$
                A Lie algebra internal to is then an object $\g \in \calV$ equipped with a so-called Lie bracket:
                    $$[-,-]: \g \tensor \g \to \g$$
                subject to two requirements:
                    \begin{enumerate}
                        \item \textbf{(Skew-symmetry)}
                            $$[-,-] + [-,-] \circ \tau_{\g, \g} = 0$$
                        \item \textbf{(The Jacobi identity)}
                            $$
                                \begin{aligned}
                                    & \left[-, [-,-]\right]
                                    \\
                                    + & \left[-, [-,-]\right] \circ \left(\id_{\g} \tensor \tau_{\g, \g}\right) \circ \left(\tau_{\g, \g} \tensor \id_{\g}\right)
                                    \\
                                    + & \left[-, [-,-]\right] \circ \left(\tau_{\g, \g} \tensor \id_{\g}\right) \circ \left(\id_{\g} \tensor \tau_{\g, \g}\right)
                                    \\
                                    = & \: 0
                                \end{aligned}
                            $$
                    \end{enumerate}
                Lie algebras internal to a symmetric monoidal $k$-linear category $\calV$ form a full subcategory which we shall denote by $\Lie\Alg(\calV)$. Its objects are the Lie algebra objects of $\g$, and its morphisms are arrows in $\calV$ that intertwine with Lie brackets, i.e. they are arrows $\phi: \g \to \h$ such that:
                    $$[-,-]_{\h} \circ (\phi \tensor \phi) = \phi \circ [-,-]_{\g}$$
                or equivalently, such that diagrams of the following form commute in $\calV$:
                    $$
                        \begin{tikzcd}
                        	{\g \tensor \g} & {\g} \\
                        	{\h \tensor \h} & {\h}
                        	\arrow["{\phi}", from=1-2, to=2-2]
                        	\arrow["{\phi \tensor \phi}"', from=1-1, to=2-1]
                        	\arrow["{[-,-]_{\h}}", from=2-1, to=2-2]
                        	\arrow["{[-,-]_{\g}}", from=1-1, to=1-2]
                        \end{tikzcd}
                    $$
            \end{definition}
            
            \begin{definition}[Enveloping algebras] \label{def: enveloping_algebras}
                Let $k$ be a ring (not necessarily commutative) and let $(\calV, \tensor, 1, \tau)$ be a symmetric monoidal $k$-linear category.
                    \begin{enumerate}
                        \item \textbf{(The Lie functor)} The Lie functor is the one that assigns to each associative and unital algebra $A$ internal to $\calV$ a Lie algebra $\frakLie(A)$ whose underlying object is just $A$, and whose Lie bracket is given by:
                            $$[-,-]_{\frakLie(A)} := \nabla_A - \nabla_A \circ \tau_{A,A}$$
                        \item \textbf{(Enveloping algebras)} Fix a Lie algebra object $\g$ of $\calV$. Then, an enveloping algebra of a Lie algebra $\g$ internal to $\calV$ is just a Lie algebra homomorphism:
                            $$e: \g \to \frakLie(A)$$
                        for some $A \in \Mon(\calV)$. These enveloping algebras form a category, whose objects are Lie algebra homomorphisms as described above, and whose morphisms are commutative triangles in $\Lie\Alg(\calV)$ as follows:
                            $$
                                \begin{tikzcd}
                                	& {\g} \\
                                	{\frakLie(A)} && {\frakLie(A')}
                                	\arrow[from=2-1, to=2-3]
                                	\arrow["{e}"', from=1-2, to=2-1]
                                	\arrow["{e'}", from=1-2, to=2-3]
                                \end{tikzcd}
                            $$
                        which we note to be induced by algebra homomorphisms $A \to A'$. We will denote the category of enveloping algebras of $\g$ by $\Env(\g)$. 
                        \\
                        The universal enveloping algebra of $\g$ (denoted by $\U(\g)$), if it exists, then its Lie algebra will be the initial object of $\Env(\g)$. 
                    \end{enumerate}
            \end{definition}
            
            \begin{theorem}[Existence and uniqueness of universal enveloping algebras] \label{theorem: universal_enveloping_algebras_universal_property}
                 Let $k$ be a ring (not necessarily commutative) and let $(\calV, \tensor, 1, \tau)$ be a symmetric monoidal $k$-linear category. Then:
                    \begin{enumerate}
                        \item \textbf{(Existence and uniqueness)} There is the following ($\calV$-enriched) adjunction if $\calV$ has all countable coproducts:
                            $$
                                \begin{tikzcd}
                                	{(\U \ladjoint \frakLie): \Mon(\calV)} & {\Lie\Alg(\calV)}
                                	\arrow["{\frakLie}"{name=0, swap}, from=1-1, to=1-2, shift right=2]
                                	\arrow["{\U}"{name=1, swap}, from=1-2, to=1-1, shift right=2]
                                	\arrow["\dashv"{rotate=-90}, from=1, to=0, phantom]
                                \end{tikzcd}
                            $$
                        wherein $\U$ is the functor sending each Lie algebra in $\calV$ to its universal enveloping algebra.
                        \item \textbf{(Explicit construction)} If $\calV$ also has all cokernels then we can explicitly characterise the universal enveloping algebra of a Lie algebra $\left(\g, [-,-]_{\g}\right)$ as the following quotient of the (Lie algebra canonically associated to) the tensor algebra $T(\g)$:
                            $$\frakLie\left(\U(\g)\right) \cong \coker \bigg(\left([-,-]_{\frakLie T(\g)} - \left(\nabla_{T(\g)} - \nabla_{T(\g)} \circ \tau_{T(\g), T(\g)}\right)\right): T(\g) \tensor T(\g) \to T(\g)\bigg)$$
                    \end{enumerate}
            \end{theorem}
                \begin{proof}
                    \noindent
                    \begin{enumerate}
                        \item We will be using Freyd's Adjoint Functor Theorem \cite[Theorem V.6.2]{maclane} to prove this assertion, and to that end, let us first note that the category $\Mon(\calV)$ is complete and locally small (this can be proven in the exact same way that one might prove that algebraic categories such as $\Ab$ or $\Ring$ are complete and locally small). Next, we will try to show that the functor $\frakLie$ satisfies the \href{https://ncatlab.org/nlab/show/solution+set+condition}{\underline{solution set condition}}, which we can do by finding a small indexing set $I$ such that for all enveloping algebras $\g \to \frakLie(A)$ of $\g$, there exists a family of algebras $A_i$ indexed by $i \in I$ and factorisations:
                            $$
                                \begin{tikzcd}
                                	& {\g} \\
                                	{\frakLie(A_i)} && {\frakLie(A)}
                                	\arrow[from=2-1, to=2-3, dashed]
                                	\arrow["{}"', from=1-2, to=2-1]
                                	\arrow["{}", from=1-2, to=2-3]
                                \end{tikzcd}
                            $$
                        (we actually do not need to worry about the size of $I$, since we have already fixed a sufficiently large Grothendieck universe). To that end, note that because $\calV$ has all countable coproducts, one can always construct tensor algebras, whose universal property implies that there is the following adjunction:
                            $$
                                \begin{tikzcd}
                                	{(T \ladjoint \oblv): \Mon(\calV)} & {\calV}
                                	\arrow["{\oblv}"{name=0, swap}, from=1-1, to=1-2, shift right=2]
                                	\arrow["{T}"{name=1, swap}, from=1-2, to=1-1, shift right=2]
                                	\arrow["\dashv"{rotate=-90}, from=1, to=0, phantom]
                                \end{tikzcd}
                            $$
                        In particular, this means that for each $A \in \Mon(\calV)$ and each Lie algebra $\g$, there is a canonical morphism from $T(\g)$ into $A$. At the same time, note that because tensor algebras are constructed as coproducts of tensor powers, one has canonical inclusions of the tensor powers $\g^{\tensor n}$ into $T(\g)$. Thus, there are commutative diagrams in $\calV$ as follows for each $A \in \Mon(\calV)$ and each $n \in \N$:
                            $$
                                \begin{tikzcd}
                                	{\g^{\tensor n}} \\
                                	& {T(\g)} & {A}
                                	\arrow[from=1-1, to=2-2]
                                	\arrow[from=2-2, to=2-3]
                                	\arrow[from=1-1, to=2-3]
                                \end{tikzcd}
                            $$
                        Thus, for each associative and unital algebra $A$, there is a universal morphism in $\Env(\g)$ as follows:
                            $$
                                \begin{tikzcd}
                                	& {\g} \\
                                	{\frakLie\left(T(\g)\right)} && {\frakLie(A)}
                                	\arrow[from=2-1, to=2-3]
                                	\arrow[from=1-2, to=2-1]
                                	\arrow[from=1-2, to=2-3]
                                \end{tikzcd}
                            $$
                        proving the existence of a small index set $I$ (the singleton in this instance) and a family of objects $\{A_i\}_{i \in I}$ of $\Mon(\calV)$ such that there are factorisations as below for all $i \in I$:
                            $$
                                \begin{tikzcd}
                                	& {\g} \\
                                	{\frakLie(A_i)} && {\frakLie(A)}
                                	\arrow[from=2-1, to=2-3, dashed]
                                	\arrow["{}"', from=1-2, to=2-1]
                                	\arrow["{}", from=1-2, to=2-3]
                                \end{tikzcd}
                            $$
                        (for the sake of clarity, let us note that here, we take $A_i = T(\g)$ for all $i \in I$). Lastly, note that the category $\Mon(\calV)$ is complete and locally small. Thus, all the conditions in Freyd's Adjoint Functor Theorem are satisfied, and therefore, $\frakLie$ is a right-adjoint. This of course means that we can construct a functor:
                            $$\U: \Lie\Alg(\calV) \to \Mon(\calV)$$
                        to be left-adjoint to $\frakLie$. Then, as a consequence of the universal property of adjoint pairs, the category of enveloping algebras of $\g$ must have $\frakLie\left(\U(\g)\right)$ as an initial object, which by definition is the so-called universal enveloping algebra of $\g$.
                        \item This is trivial if we note that taking the quotient of $T(\g)$ by the equivalence relation generated by the image of $[-,-]_{\frakLie T(\g)} - \left(\nabla_{T(\g)} - \nabla_{T(\g)} \circ \tau_{T(\g), T(\g)}\right)$ (which incidentally, also canonically endows $T(\g)$ with a Lie bracket) is just to ensure that the canonical map:
                            $$\g \to (\frakLie \circ \U \circ \frakLie \circ T)(\g)$$
                        is a Lie algebra homomorphism for every $\g$.
                    \end{enumerate}
                \end{proof}
            \begin{corollary}
                $\frakLie$ is left-exact and $\U$ is right-exact. In particular, we have:
                    $$\U\left(\g \oplus \g'\right) \cong \U(\g) \tensor \U(\g')$$
                for any pair $\g, \g'$ of Lie algebras. 
            \end{corollary}
                \begin{proof}
                    These are general properties of adjoint functors.
                \end{proof}
            \begin{remark}
                The adjunction as presented in the preceding theorem can be understood as fitting into the following (non-commutative) diagram:
                    $$
                        \begin{tikzcd}
                        	{\Mon(\calV)} & {} & {\calV} \\
                        	& {\Lie\Alg(\calV)}
                        	\arrow["{\oblv}"{name=0, swap}, from=1-1, to=1-3, shift right=2]
                        	\arrow["{T}"{name=1, swap}, from=1-3, to=1-1, shift right=2]
                        	\arrow["{L}"{name=2, swap}, from=2-2, to=1-3, shift right=5]
                        	\arrow[""{name=3, inner sep=0}, from=1-3, to=2-2, shift left=1]
                        	\arrow["{\U}"{name=4, swap}, from=2-2, to=1-1, shift left=1]
                        	\arrow["{\frakLie}"{name=5, swap}, from=1-1, to=2-2, shift right=5]
                        	\arrow["\dashv"{rotate=-90}, from=1, to=0, phantom]
                        	\arrow["\dashv"{rotate=61}, from=5, to=4, phantom]
                        	\arrow["\dashv"{rotate=129}, from=2, to=3, phantom]
                        \end{tikzcd}
                    $$
            \end{remark}
            \begin{remark}[Universal enveloping algebras are bialgebras] \label{remark: universal_enveloping_algebras_are_bialgebras}
                Let $k$ be a ring and let $\g$ be a Lie algebra internal to some $k$-linear symmetric monoidal category $(\V, \tensor, 1, \tau)$. Then, its universal enveloping algebra $\U(\g)$ is a cocommutative bialgebra internal to $\V$, which is commutative if and only if $\g$ is abelian. 
                
                This becomes trivial if we let the comultiplication be given by:
                    $$\Delta_{\U(\g)} := \id_{\U(\g)} \tensor \e + \e \tensor \id_{\U(\g)}$$
                and the counit be:
                    $$\e_{\U(\g)} := 0$$
            \end{remark}
                
            \begin{lemma}[Embedding of Lie algebras into their universal enveloping algebras]
                Let $k$ be a ring (not necessarily commutative) and let $(\calV, \tensor, 1, \tau)$ be a symmetric monoidal $k$-linear \textbf{abelian} category. Also, suppose that $\g$ is a (faithfully ?) flat Lie algebra object internal to $\calV$, i.e. that the functor $\g \tensor -$ is (faithfully ?) flat. Then, there is a canonical monomorphism embedding $\g$ into $\frakLie \U(\g)$
            \end{lemma}
                \begin{proof}
                    \todo[inline]{Currently I'm not certain that this statement is entirely correct.}
                \end{proof}
            
            \begin{example}[The abelian case]
                If $\g$ is an abelian Lie algebra (i.e. one whose Lie bracket is just the zero morphism), then as a direct consequence of the definition of symmetric algebras and of theorem \ref{theorem: universal_enveloping_algebras_universal_property}, we have the following isomorphism of associative and unital algebras:
                    $$\U(\g) \cong \Sym(\g)$$
            \end{example}
            \begin{example}[The Poincar\'e-Birkhoff-Witt Theorem]
                Let $k$ be a commutative and unital $\Q$-algebra, let $(\calV, \tensor, 1, \tau)$ be a symmetric monoidal $k$-linear category with all countable coproducts and cokernels, and let $\g$ be a Lie algebra object of $\calV$ that is (faithfully ?) flat (i.e. one such that the functor $\g \tensor -$ is left-exact). Then, there is the following isomorphism of cocommutative $\N$-graded $k$-bialgebras:
                    $$\U(\g) \cong \Sym(\g)$$
                In other words, one can understand representations of such a Lie algebra $\g$ via representations of the symmetric algebra $\Sym(\g)$, which in turn are just representations of symmetric groups.
            \end{example}
            \begin{example}[A counter-example]
                Let $k$ be a field of characteristic $0$ and consider the symmetric monoidal category of finite-dimensional $k$-vector spaces. Then clearly, $\calV$ does not have all countable coproducts (for example, the vector space $k^{\oplus \aleph_0}$, which is a coproduct indexed by the countable infinite cardinal $\aleph_0$, is not an object as it is infinite-dimensional), and thus not all Lie $k$-algebras have a universal enveloping algebra.
            \end{example}
            \begin{example}[Open problem]
                It is not known if the functor:
                    $$\U: \frakLie(\calV) \to \Lie\Alg(\calV)$$
                is faithful, and even if that is not the case in general, we also do not have a good understanding of which extra assumptions to impose on our ambient symmetric monoidal linear category $\calV$ so that in such a setting, $\U$ might be faithful. 
            \end{example}
            
            \begin{lemma}[Tannaka duality] \label{lemma: tannaka_duality}
                Let $\calV$ be a closed symmetric monoidal category that is locally small (such as the symmetric monoidal category of vector spaces over a field), so that it may be enriched over itself via its internal homs (which exist thanks to the monoidal closure assumption), and let $A$ be a monoid object of $\calV$. If we denote the category of $\calV$-representations on $A$ by:
                    $$\Rep_{\calV}(A) := \calV\-\Cat(\bfB A^{\op}, \calV)$$
                then the algebra $\End_{\calV\-\Cat(\Rep_{\calV}(A), \calV)}(F)$ of $\calV$-natural endomorphisms on the canonical forgetful functor $F: \Rep_{\calV}(A) \to \calV$ is isomorphic to $A$.
            \end{lemma}
                \begin{proof}
                    Apply the $\calV$-enriched Yoneda's lemma:
                        $$
                            \begin{aligned}
                                \End_{\calV\-\Cat(\Rep_{\calV}(A), \calV)}(F) & \cong \calV\-\Cat(\Rep_{\calV}(A), \calV)(F, F)
                                \\
                                & \cong \Psh_{\calV}\left(\Psh_{\calV}(\bfB A)\right)\bigg(\Psh_{\calV}(\bfB A)(*, -), \Psh_{\calV}(\bfB A)(*, -)\bigg)
                                \\
                                & \cong \Psh_{\calV}(\bfB A)(*, *)
                                \\
                                & \cong \Rep_{\calV}(A)(*, *)
                                \\
                                & \cong \End_{\calV}(A)
                                \\
                                & \cong A
                            \end{aligned}
                        $$
                \end{proof}
            \begin{theorem}
                If $k$ is a ring, $\calV$ is a \textit{locally small} \textit{closed} symmetric monoidal $k$-linear category with all coproducts and cokernels, and $\g$ is a (faithfully ?) flat Lie algebra object of $\calV$, then there is an equivalence of abelian monoidal $k$-linear categories as follows:
                    $$\Rep_{\calV}(\g) \cong {\U(\g)}\mod$$
                wherein $\Rep_{\calV}(\g)$, the category of representations of $\g$ on $\calV$, is the category whose objects are Lie algebra homomorphisms from $\g$ to $\frakgl(V)$ (with $\frakgl(V)$ the canonical Lie algebra associated to the associative and unital endomorphism algebra $\End_{\calV}(V)$), and morphisms are commutative triangles in $\Lie\Alg(\calV)$ of the form:
                    $$
                        \begin{tikzcd}
                        	& {\g} \\
                        	{\frakgl(V)} && {\frakgl(V')}
                        	\arrow[from=2-1, to=2-3]
                        	\arrow[from=1-2, to=2-1]
                        	\arrow[from=1-2, to=2-3]
                        \end{tikzcd}
                    $$
                Also, note that ${\U(\g)}\mod$ is \href{https://ncatlab.org/nlab/show/module+over+a+monoid}{\underline{well-defined}} as the category of $\U(\g)$-left-equivariant objects of $\calV$, and it exhibits all properties that one might expect of a module category.
            \end{theorem}
                \begin{proof}
                    Before we prove this claim, let us first note that because $\calV$ has all countable coproducts and cokernels, all Lie algebras possess universal enveloping algebras. With that out of the way, let us note that by Tannaka duality (cf. lemma \ref{lemma: tannaka_duality}), each left-$\U(\g)$-module is actually just a $\U(\g)$-representation. Thus, to show that:
                        $$\Rep_{\calV}(\g) \cong {\U(\g)}\mod$$
                    it will suffice to show that:
                        $$\Rep_{\calV}(\g) \cong \Rep_{\calV}\left(\U(\g)\right)$$
                    (this might seem like a round-about method, but without Tannaka duality, guaranteeing that the equivalence is actually between abelian monoidal $k$-linear categories instead of just between ordinary categories will be difficult). In turn, one can do this via showing that there is the following equivalence of categories of Lie algebra representations:
                        $$\Rep_{\calV}(\g) \cong \Rep_{\calV}\left(\frakLie \U(\g)\right)$$
                    thanks to the uniqueness of the universal enveloping algebra. Then, we can simply apply lemma 1.1.1, which states that $\g$ is a subobject of $\frakLie \U(\g)$, and consider commutative diagrams as follows in $\Lie\Alg(\calV)$:
                        $$
                            \begin{tikzcd}
                            	{\g} \\
                            	& {\frakgl(V)} \\
                            	{\frakLie \U(\g)}
                            	\arrow[from=1-1, to=2-2]
                            	\arrow[from=3-1, to=2-2]
                            	\arrow[from=1-1, to=3-1, tail]
                            \end{tikzcd}
                        $$
                    to show that for each representation of $\g$ on $V \in \calV$, there is a representation of $\frakLie \U(\g)$ on $V$ as well, and vice versa.
                \end{proof}
                
        \subsubsection{Cohomologies and homologies of Lie algebras}
            \begin{convention}
                From now on, we work within a fixed $k$-linear tensor category $\calV$ (i.e. a $k$-linear abelian dualisable symmetric monoidal category). For all intents and purposes, one might think of $\calV$ as being $k\mod$.
            \end{convention}
            
            We begin by proving that the category $\Lie\Alg(\calV)$ has zero objects, kernels, and cokernels.
            \begin{proposition}[The zero Lie algebra] \label{prop: zero_lie_algebra}
                The zero object $0 \in \calV$ is also the zero object in $\Lie\Alg(\calV)$.
            \end{proposition}
                \begin{proof}
                    One simply has to note that there is only one unique morphism from $0$ to any Lie algebra $\g$, and likewise, from any Lie algebra $\g$ to $0$. 
                \end{proof}
                
            \begin{proposition}[(Co)kernels of Lie algebra homomorphisms] \label{prop: (co)kernels_of_lie_algebra_homomorphisms}
                For any Lie algebra homomorphism $\phi: \g \to \h$, the following pullback and pushout exist in $\Lie\Alg(\calV)$:
                    $$
                        \begin{tikzcd}
                        	{\ker \phi} & 0 \\
                        	\g & \h
                        	\arrow["\phi", from=2-1, to=2-2]
                        	\arrow[from=1-1, to=2-1]
                        	\arrow[from=1-1, to=1-2]
                        	\arrow[from=1-2, to=2-2]
                        \end{tikzcd}
                    $$
                    $$
                        \begin{tikzcd}
                        	\g & \h \\
                        	0 & {\coker \phi}
                        	\arrow["\phi", from=1-1, to=1-2]
                        	\arrow[from=1-1, to=2-1]
                        	\arrow[from=2-1, to=2-2]
                        	\arrow[from=1-2, to=2-2]
                        	\arrow["\lrcorner"{anchor=center, pos=0.125, rotate=180}, draw=none, from=2-2, to=1-1]
                        \end{tikzcd}
                    $$
            \end{proposition}
                \begin{proof}
                    We can embed $\calV$ into a category of modules over a commutative ring, \textit{\`a la} Freyd-Mitchell, so for the purposes of this proof, we might as well take $\calV \cong k\mod$. Our strategy is then to compute $\ker \phi$ and $\coker \phi$ in $k\mod$, and then verify that they have natural Lie algebra structures. 
                    
                    To this end, consider first of all the folloiwng:
                        $$\ker \phi := \{x \in \g \mid \phi(x) = 0\}$$
                    Since $\ker \phi$ is a $k$-submodule of $\g$, we claim that the Lie bracket on $\ker \phi$ is precisely that of $\g$. To verify that this is true, simply consider:
                        $$\phi([x, y]) = [\phi(x), \phi(y)] = [0, 0] = 0$$
                    wherein $x, y \in \g$ are arbitrary elements. 
                    
                    Now, consider:
                        $$\coker \phi := \h/\im\phi$$
                \end{proof}
            
            Now, we shall define Lie subalgebras and Lie ideals. 
            \begin{definition}[Subalgebras and ideals] \label{def: lie_subalgebras_and_lie_ideals}
                Let $\g$ be a Lie algebra internal to $\calV$. 
                    \begin{enumerate}
                        \item \textbf{(Subalgebras):} A \textbf{Lie subalgebra} of $\g$ is a monomorphism of Lie algebras\footnote{Note that this need not be a kernel, because only \say{$k$-linear} morphisms in the ambient tensor category $\calV$ are kernel if and only if they are monomorphisms; the same need not be true for Lie algebra homomorphisms.}.
                        \item \textbf{(Ideals):} A \textbf{Lie algebra ideal} is not just a monomorphism, but futhermore, the kernel of a Lie algebra homomorphism (i.e. it is a \textit{normal} monomorphism\footnote{Note how this is in perfect analogy with normal subgroups, which are precisely kernels of group homomorphisms.}). 
                    \end{enumerate}
            \end{definition}
        
            \begin{definition}[Lie algebra cohomologies] \label{def: lie_algebra_cohomologies}
                Let $\g$ be a Lie algebra over a commutative ring $k$ and let $A$ be a $\g$-module. Then, we define the $n^{th}$ cohomology group of $\g$ with coefficient in $A$ as:
                    $$H^n(\g, A) \cong \Ext^n_{\U(\g)}(k, A)$$
                where we view $k$ as the $1$-dimensional $\g$-representation on the right-hand side. 
            \end{definition}
            \begin{remark}
                It is not hard to see via induction on the cohomological dimension $n \in \N$ that $H^n(\g, A)$ (as in definition \ref{def: lie_algebra_cohomologies}) actually has a $k$-module structure.
            \end{remark}
            
            Let us now attempt to compute cohomologies of Lie algebras in low dimensions (namely, $n = 0, 1, 2$) as well as give meaning to these spaces (cf. proposition \ref{prop: low_dimensional_cohomologies_of_lie_algebras}). We shall require, first of all, the following lemma:
            \begin{lemma}[de Rham resolutions of Lie algebras] \label{lemma: de_rham_resolutions_of_lie_algebras}
                For $\g$ a Lie algebra over a commutative ring $k$ and $A$ any $\g$-module, one has the following isomorphism of cohomology groups:
                    $$H^n(\g, A) \cong \Ext^n_k(\Lambda^{\bullet}(\g), A)$$
            \end{lemma}
                \begin{proof}
                    By definition, we have:
                        $$H^n(\g, A) \cong \Ext^n_{\U(\g)}(k, A)$$
                    Through noting that $\Lambda^{\bullet}(\g) \cong \{k \to \g \to \g \wedge \g \to \cdots\}$ is an injective resolution of $k \in \U(\g)\mod$, one sees that these isomorphisms of cohomologies \say{lift} to the following quasi-isomorphism of cochain complexes:
                        $$H^{\bullet}(\g, A) \cong_{\qis} \R\Hom_{\U(\g)}\left(\U(\g) \tensor_k^{\L} \Lambda^{\bullet}(\g), A\right)$$
                    An application of the tensor-hom adjunction then gives:
                        $$H^{\bullet}(\g, A) \cong_{\qis} \R\Hom_k(\Lambda^{\bullet}(\g), A)$$
                    which implies:
                        $$H^n(\g, A) \cong \Ext^n_k(\Lambda^{\bullet}(\g), A)$$
                \end{proof}
            
            Now, because the low-dimensional cohomology groups of Lie algebras are interpreted in terms of various extra structures on these Lie algebras, let us first write down some auxiliary definitions.
            \begin{definition}[Derivations on Lie algebras] \label{def: lie_algebra_derivations}
                A \textbf{derivation} on a Lie algebra $\g$ internal to any suitable tensor category (cf. definition \ref{def: lie_algebras}) is nothing but a derivation, i.e. an endomorphism:
                    $$D: \g \to \g$$
                making the following diagram (expressing the Leibniz Rule) commute:
                    $$
                        \begin{tikzcd}
                        	{\g \tensor \g} & \g \\
                        	{\g \tensor \g} & {\g}
                        	\arrow["{D \tensor \id_{\g} + \id_{\g} \tensor D}"', from=1-1, to=2-1]
                        	\arrow["{[-,-]}", from=2-1, to=2-2]
                        	\arrow["{[-,-]}", from=1-1, to=1-2]
                        	\arrow["D", from=1-2, to=2-2]
                        \end{tikzcd}
                    $$
                Here, 
            \end{definition}
            \begin{remark}
                The space of derivations on a given Lie algebra $\g$ over a commutative ring $k$ is also a Lie algebra over $k$, whose Lie bracket is inherited from the endomorphism algebra $\frakgl(\g)$. We denote this Lie algebra by $\der(\g)$.
            \end{remark}
            \begin{proposition}[Lie algebra derivation universal property] \label{prop: lie_algebra_derivations_universal_property}
                Let $\g$ be a Lie algebra over a commutative ring $k$\footnote{One can generalise this result rather easily to a setting wherein $\g$ is interal to a tensor category: $k$ can simply be replace by the monoidal unit.}. Then, $\der(\g, -): \g\mod \to k\mod$ is not only a functor, but also left-adjoint to the forgetful functor $\oblv: k\mod \to \g\mod$. Furthermore, it is corepresented by the augmentation ideal:
                    $$\rmI(\g) := \ker(\Delta_{\U(\g)}: \U(\g) \tensor \U(\g) \to \U(\g): f \tensor g \mapsto fg)$$
                i.e.:
                    $$\der(\g, -) \cong \Hom_{(\g)}(\rmI(\g), -)$$
            \end{proposition}
                \begin{proof}
                    
                \end{proof}
            \begin{definition}[Inner derivations on Lie algebras] \label{def: lie_algebra_inner_derivations}
                Any derivation $D: \g \to \g$ on a Lie algebra $\g$ such that:
                    $$\forall y \in \g: \exists x \in \g: D(y) = \ad_{\g}(x)(y) = [x, y]$$
                is known as an \textbf{inner derivation}\footnote{This is in analogy with inner automorphisms of groups}.
            \end{definition}
            \begin{remark}[Adjoint maps are derivations]
                It is an easy exericse (one can simply manipulate the Jacobi identity defining Lie brackets) to show that any adjoint map:
                    $$[x, -]: \g \to \g$$
                is actually a derivation. We thus leave it up to the reader to fill in the details. 
            \end{remark}
            \begin{convention}
                The set of inner derivations on a given Lie algebra $\g$ over a commutative ring $k$ is a Lie algebra over $k$ whose structure is inherited from the endomorphism algebra $\frakgl(\g)$ (namely, its Lie bracket is the commutator with respect to composition of endomorphisms). We denote this Lie algebra by $\inn(\g)$.
            \end{convention}
            \begin{remark}
                $\inn(\g)$ is a Lie subalgebra of $\der(\g)$ (cf. definition \ref{def: lie_subalgebras_and_lie_ideals}), and hence of $\frakgl(\g)$. 
            \end{remark}
                
            We can now compute the low-dimensional cohomology groups of Lie algebras.
            \begin{proposition}[Low-dimensional cohomologies of Lie algebras] \label{prop: low_dimensional_cohomologies_of_lie_algebras}
                For $\g$ a Lie algebra over a commutative ring $k$ and $A$ any $\g$-module, one has the following interpretations of the low-dimensional cohomology spaces of $\g$ with coeffcients in $A$:
                \begin{enumerate}
                    \item \textbf{($n = 0$: Invariants):} There exists a canonical $k$-module isomorphism:
                        $$H^0(\g, A) \cong \{a \in A \mid \forall x \in \g: x \cdot a = 0\}$$
                    When $A \cong \g$, this is furthermore an isomorphism of Lie algebras over $k$, with the Lie bracket on the left-hand side being inherited from $\g$.
                    \item \textbf{($n = 1$: Derivations):} There exists a canonical Lie algebra isomorphism:
                        $$H^1(\g, \g) \cong \out(\g)$$
                    \item \textbf{($n = 2$: Extensions):} There exists a natural $k$-module isomorphism between the space of isomorphism classes of extensions of $\g$ by $A$ and $H^2(\g, A)$.
                \end{enumerate}
            \end{proposition}
                \begin{proof}
                    \noindent
                    \begin{enumerate}
                        \item \textbf{($n = 0$: Invariants):}
                        \item \textbf{($n = 1$: Derivations):}
                        \item \textbf{($n = 2$: Extensions):}
                    \end{enumerate}
                \end{proof}
            
            Lemma \ref{lemma: de_rham_resolutions_of_lie_algebras} also help us makes sense of homologies of Lie algebras, defined as follows:
            \begin{definition}[Lie algebra homologies] \label{def: lie_algebra_homologies}
                Let $\g$ be a Lie algebra over a commutative ring $k$ and let $A$ be a $\g$-module. Then, we define the $n^{th}$ cohomology group of $\g$ with coefficient in $A$ as:
                    $$H_n(\g, A) \cong \Tor_n^{\U(\g)}(k, A)$$
                where we view $k$ as the $1$-dimensional $\g$-representation on the right-hand side. 
            \end{definition}
            \begin{remark}
                It is not hard to show, using the associativity of the derived tensor product, that one has the following quasi-isomorphism:
                    $$\Tor_{\bullet}^{\U(\g)}(k, A) \cong_{\qis} \Lambda^{\bullet}(\g) \tensor_k^{\L} A$$
            \end{remark}
        
            \begin{definition}[Simple Lie algebras] \label{def: simple_lie_algebras}
                A Lie algebra internal to an appropriate tensor category $\calV$ (cf. definition \ref{def: lie_algebras}) is \textbf{simple} if and only if it is a \textit{non-abelian} (i.e. the Lie bracket in question is not the zero morphism) simple object of $\Lie\Alg(\calV)$.
                
                More algebraically, one might say that a simple Lie algebra is a non-abelian Lie algebra with no non-trivial proper ideal.
            \end{definition}
        
            Let us now move on to two important results concerning semi-simple Lie algebras, namely:
                \begin{itemize}
                    \item the fact that finite-dimensional representations of semi-simple Lie algebras are completely reducible, and
                    \item that every Lie algebra which is finite-dimensional splits into the direct sum of its so-called \say{radical} and some semi-simple Lie algebra. 
                \end{itemize}
            During the way, we shall prove two technical lemmas, commonly known as Whitehead's First and Second Lemmas.
            
            \begin{convention} \label{conv: cohomology_of_semi_simple_lie_algebras_conventions}
                Throughout this paragraph, $\g$ shall denote a finite-dimensional semi-simple Lie algebra over a field $k$ of characteristic $0$, and $A$ shall denote a $\g$-module that is of finite dimension as a $k$-vector space (i.e. a finite-dimensional $\g$-representation) with structural homomorphism:
                    $$\rho: \g \to \frakgl(A)$$
            \end{convention}
            
            \begin{proposition}[Associated invariant bilinear forms] \label{prop: associated_invariant_bilinear_forms_of_lie_algebras}
                To $\g$, $A$, and $\rho$ as in convention \ref{conv: cohomology_of_semi_simple_lie_algebras_conventions}, there exists a symmetric bilinear form:
                    $$\beta: \g \tensor \g \to k$$
                given by\footnote{Note that $\beta(x, y)$ should technically be written as $\beta(x \tensor y)$.}:
                    $$x \tensor y \mapsto \beta(x, y) := \trace(\rho(x) \rho(y))$$
                Furthermore, such a bilinear form is $\ad_{\g}$-invariant, i.e.:
                    $$\beta([x, y], z) = \beta(x, [y, z])$$
                for all $x, y, z \in \g$.
            \end{proposition}
                \begin{proof}
                    Bilinearity is a straightforward consequence of the linearity of $\rho$ (in the two factors $x$ and $y$) and the linearity of the trace map $\trace: \frakgl(A) \to k$. As for symmetry, it is a consequence of the well-known fact that traces of endomorphisms on finite-dimensional vector spaces (whihc $\frakgl(A)$ is, since $A$ is a finite-dimensional $k$-vector space) are invariant under cyclic permutations.
                    
                    Now, to prove that $\beta$ is $\ad_{\g}$-invariant, consider the following:
                        $$
                            \begin{aligned}
                                \beta([x, y], z) & = \trace(\rho([x, y]) \rho(z))
                                \\
                                & = \trace([\rho(x), \rho(y)] \rho(z))
                                \\
                                & = \trace((\rho(x)\rho(y) - \rho(y)\rho(x)) \rho(z))
                                \\
                                & = \trace(\rho(x) \rho(y)\rho(z)) - \trace(\rho(y)\rho(x)\rho(z))
                                \\
                                & = \trace(\rho(x) \rho(y)\rho(z)) - \trace(\rho(x)\rho(z)\rho(y))
                                \\
                                & = \trace(\rho(x)[\rho(y), \rho(z)])
                                \\
                                & = \beta(x, [y, z])
                            \end{aligned}
                        $$
                \end{proof}
            \begin{example}[The Killing Form] \label{example: the_killing_form}
                Since $\g$ is assumed to be finite-dimensional over $k$, one can meaningfully construct ymmetric bilinear forms in the fashion of proposition \ref{prop: associated_invariant_bilinear_forms_of_lie_algebras} for the case $A \cong \g$. In particular, when $\rho$ is the adjoint representation:
                    $$\ad_{\g}: \g \to \End_k(\g): x \mapsto \left(x \mapsto [x, -]: \g \to \g\right)$$
                the associated symmetric bilinear form is the \textbf{Killing Form} $\kappa: \g \tensor \g \to k$, which is defined via:
                    $$\kappa(x, y) := \trace(\ad_{\g}(x) \ad_{\g}(y))$$
            \end{example}
            
            \begin{theorem}[Non-degeneracy of associated invariant bilinear forms] \label{theorem: nondegeneracy_of_associated_invariant_bilinear_forms_of_lie_algebras}
                The associated invariant bilinear form is non-degenerate when the representation $\rho: \g \to \frakgl(A)$ is faithful.
            \end{theorem}
                \begin{proof}
                    
                \end{proof}
            \begin{corollary}
                The adjoint representation is faithful, so the Killing Form is non-degenerate. 
            \end{corollary}
                
        \subsubsection{Structure and classification of compact Lie algebras}

    \subsection{Localisation of \texorpdfstring{$\g$}{}-modules; the Beilinson-Bernstein Equivalence}
        \begin{convention} \label{conv: beilinson_bernstein_localisation_conventions}
            \noindent
            \begin{itemize}
                \item We work with a complex algebraic group $G$ of adjoint type with \textit{a priori} simple Lie algebra $\g$, along with universal enveloping algebra $\U(\g)$ and centre $\rmZ(\g)$ thereof. $B$ shall be a Borel subgroup thereof. 
                \item $\Gr_G$ shall denote the loop affine Grassmannian $G(\!(t)\!)/G[\![t]\!]$ attached to $G$.
                \item If $T$ is a maximal torus inside $G$ and $\lambda \in \bbX(T)$ is a weight then we shall denote by $\Dmod(G/B)^{(\lambda)}$ the category of $\lambda$-twisted D-modules on $G/B$ (i.e. D-modules on $G/B$ which act on the line bundle $G \x_B \lambda$), where $B \leq G$ is a choice of Borel subgroup that contains the fixed torus $T$.
                \item The Lie algebras of $G, B$, and $T$ shall be denoted - respectively - by $\g, \b$, and $\t$.
            \end{itemize}
        \end{convention}
        
        Let $\pi: G \to G/B$ denote the canonical projection and observe that there exists a canonically determined sheaf pullback functor:
            $$\pi^*: \Dmod(G/B) \to \Dmod(G)$$
        which in turn induces a \say{twisted} pullback functor, compatible with the aforementioned pullback $\pi^*$ in some appropriate and natural sense:
            $$(\pi^{(\lambda)})^*: \Dmod(G/B)^{(\lambda)} \to \Dmod(G)$$
        Also, note the fact that for all $\calM \in \Dmod(G)$, the global section $\Gamma(G, \calM)$ has a natural $\U(\g)$-bimodule structure thanks to the self-actions of $G$ via left and right-translations. Among other things, this implies that there exists the following natural composition of functors:
            $$
                \begin{tikzcd}
                	{\Dmod(G/B}) & {\Dmod(G)} & {\U(\g)\bimod}
                	\arrow["{\pi^*}", from=1-1, to=1-2]
                	\arrow["{\Gamma(G, -)}", from=1-2, to=1-3]
                \end{tikzcd}
            $$
        which induces the following composition wherein the first functor is now \say{twisted} by the character $\lambda$:
            $$
                \begin{tikzcd}
                	{\Dmod(G/B)^{(\lambda)}} & {\Dmod(G)} & {\U(\g)\bimod}
                	\arrow["{(\pi^{(\lambda)})^*}", from=1-1, to=1-2]
                	\arrow["{\Gamma(G, -)}", from=1-2, to=1-3]
                \end{tikzcd}
            $$
        Set:
            $$\Gamma^{(\lambda)} := \Gamma(G, -) \circ (\pi^{(\lambda)})^*$$
        
        Now, let $\V_{\lambda}$ denote the Verma module attached to a given weight $\lambda$ and set:
            $$\bfGamma^{(\lambda)} := \Hom_{\U(\g)}(\V_{\lambda}, -) \circ \Gamma^{(\lambda)}$$
        Eventually, we will be establishing an adjunction:
            $$
                \begin{tikzcd}
                	{\Dmod(G/B)^{(\lambda)}} & \calV
                	\arrow[""{name=0, anchor=center, inner sep=0}, "{\bfGamma^{(\lambda)}}"', shift right=2, from=1-1, to=1-2]
                	\arrow[""{name=1, anchor=center, inner sep=0}, "{\Delta^{(\lambda)}}"', shift right=2, from=1-2, to=1-1]
                	\arrow["\dashv"{anchor=center, rotate=-90}, draw=none, from=1, to=0]
                \end{tikzcd}
            $$
        wherein $\calV$ is the so-called \say{Category $\O$} whose construction we shall get to\footnote{For now, think of $\calV$ as a suitable subcategory of the category of right-$\U(\g)$-modules.}; here $\Delta^{(\lambda)}$ is given by:
            $$M \mapsto \D_{G/B} \tensor^{\L}_{\U(\g)} M$$
            
        All of this is simply to say that, the global section of the pullback along $\pi: G \to G/B$ of any D-modules on $G/B$ twisted by some character $\lambda$, remains unchanged under \say{twisting} by the Verma module $\V_{\lambda}$. The adjunction $\Delta^{(\lambda)} \ladjoint \bfGamma^{(\lambda)}$ - as complicated as it may seem - is actually just an enhanced version of the usual tensor-hom adjunction.
        
        Now, this is not all there is to the theorem; it in fact says a lot more. Interesting phenomena occur when the weight $\lambda + \rho$ (where $\rho$ is the half-sum of the positive roots of $\g$) is \textit{dominant}: specifically, should this be the case, the Verma module $\V_{\lambda}$ shall be a \textit{projective} object of the category $\calV$, and hence the \say{twisted} global section functor:
            $$\bfGamma^{(\lambda)}: \Dmod(G/B)^{(\lambda)} \to \calV$$
        shall be the composition of two exact functors, and thus \textit{exact} itself. One can interpret this fact as $G/B$ being \say{D-affine} up to a twisting by a weight, in the same sense that higher cohomologies of quasi-coherent sheaves over Noetherian affine schemes vanish. 
                
        \subsubsection{Kazhdan-Lusztig modules, the category \texorpdfstring{$\calV$}{}, and Verma modules}
                
        \subsubsection{Proving the theorem}
            The following geometric result shall help us make better sense of the Beilinson-Bernstein Localisation Theorem (cf. theorem \ref{theorem: beilinson_bernstein_localisation}). In fact, it is the true \say{localisation theorem}: the Beilinson-Bernstein Theorem simply refines the adjunction established by this results down to an adjoint equivalence (which, of course, is of great representation-theoretic significance)\footnote{Additionally, note how the notion of weight does not appear at all in theorem \ref{theorem: localisation_adjunction_for_D_modules}}.
            
            \begin{lemma}[The localisation adjunction for quasi-coherent associative algebras] \label{lemma: localisation_adjunction_for_quasi_coherent_associative_algebras}
                Let $X$ be a scheme and consider $\calB \in \Assoc\Alg(\QCoh(X))$ with global section $B \cong \Gamma(X, \calB)$. Then, one has the following derived adjunction:
                    $$
                        \begin{tikzcd}
                        	{{}^l\calB\mod} & {B\bimod}
                        	\arrow[""{name=0, anchor=center, inner sep=0}, "{\R\Gamma(X, -)}"', shift right=2, from=1-1, to=1-2]
                        	\arrow[""{name=1, anchor=center, inner sep=0}, "{\calB \tensor_B^{\L} -}"', shift right=2, from=1-2, to=1-1]
                        	\arrow["\dashv"{anchor=center, rotate=-90}, draw=none, from=1, to=0]
                        \end{tikzcd}
                    $$
            \end{lemma}
                \begin{proof}
                    By definition, $\R\Gamma(X, -) \cong \R\Hom_{{}^l\calB\mod}(\calB, -)$, so this is just the hom-tensor adjunction.
                \end{proof}
            \begin{corollary} \label{coro: localisation_adjunction_for_quasi_coherent_associative_algebras}
                If $B$ (as in lemma \ref{lemma: localisation_adjunction_for_quasi_coherent_associative_algebras}) is a bialgebra over some \say{deeper} base associative ring $B_0$ then the adjunction of lemma \ref{lemma: localisation_adjunction_for_quasi_coherent_associative_algebras} will extend down to:
                    $$
                        \begin{tikzcd}
                        	{{}^l\calB\mod} & {B_0\bimod}
                        	\arrow[""{name=0, anchor=center, inner sep=0}, "{\R\Gamma(X, -)}"', shift right=2, from=1-1, to=1-2]
                        	\arrow[""{name=1, anchor=center, inner sep=0}, "{\calB \tensor_{B_0}^{\L} -}"', shift right=2, from=1-2, to=1-1]
                        	\arrow["\dashv"{anchor=center, rotate=-90}, draw=none, from=1, to=0]
                        \end{tikzcd}
                    $$
            \end{corollary}
            \begin{theorem}[The localisation adjunction for D-modules] \label{theorem: localisation_adjunction_for_D_modules}
                Let $X$ be a smooth algebraic variety with a $G$-action\footnote{$G$ still as in convention \ref{conv: beilinson_bernstein_localisation_conventions}} (which means that $G$ \textit{a priori} has the structure of an $X$-scheme through a structural morphism $\pi: G \to X$). Then, there exists the following derived adjunction:
                    $$
                        \begin{tikzcd}
                        	{\Dmod(X)} & {\U(\g)\bimod}
                        	\arrow[""{name=0, anchor=center, inner sep=0}, "{\bfGamma}"', shift right=2, from=1-1, to=1-2]
                        	\arrow[""{name=1, anchor=center, inner sep=0}, "{\D_X \tensor^{\L}_{\U(\g)} -}"', shift right=2, from=1-2, to=1-1]
                        	\arrow["\dashv"{anchor=center, rotate=-90}, draw=none, from=1, to=0]
                        \end{tikzcd}
                    $$
                wherein the (derived) functor $\bfGamma: \Dmod(X) \to \U(\g)\bimod$ returns global sections of pullbacks of D-modules on $X$ to $G$, i.e. for all $\calM \in \Dmod(X)$, one has:
                    $$\bfGamma(\calM) \cong \Gamma(G, \pi^*\calM)$$
            \end{theorem}
                \begin{proof}
                    Corollary \ref{coro: localisation_adjunction_for_quasi_coherent_associative_algebras} tells us that there exists the following adjunction:
                        $$
                            \begin{tikzcd}
                            	{\Dmod(G)} & {\U(\g)\bimod}
                            	\arrow[""{name=0, anchor=center, inner sep=0}, "{\R\Gamma(G, -)}"', shift right=2, from=1-1, to=1-2]
                            	\arrow[""{name=1, anchor=center, inner sep=0}, "{\D_G \tensor^{\L}_{\U(\g)} -}"', shift right=2, from=1-2, to=1-1]
                            	\arrow["\dashv"{anchor=center, rotate=-90}, draw=none, from=1, to=0]
                            \end{tikzcd}
                        $$
                    Next, consider the following pair of derived adjunctions:
                        $$
                            \begin{tikzcd}
                            	{\Dmod(X)} & {\Dmod(G)} & {\U(\g)\bimod}
                            	\arrow[""{name=0, anchor=center, inner sep=0}, "{\R\pi^*}"', shift right=2, from=1-1, to=1-2]
                            	\arrow[""{name=1, anchor=center, inner sep=0}, "{\D_X \tensor^{\L}_{\D_G} -}"', shift right=2, from=1-2, to=1-1]
                            	\arrow[""{name=2, anchor=center, inner sep=0}, "{\R\Gamma(G, -)}"', shift right=2, from=1-2, to=1-3]
                            	\arrow[""{name=3, anchor=center, inner sep=0}, "{\D_G \tensor^{\L}_{\U(\g)} -}"', shift right=2, from=1-3, to=1-2]
                            	\arrow["\dashv"{anchor=center, rotate=-90}, draw=none, from=1, to=0]
                            	\arrow["\dashv"{anchor=center, rotate=-90}, draw=none, from=3, to=2]
                            \end{tikzcd}
                        $$
                    They are trivially composable, and the resulting pair of functors is \textit{a priori} the sought-for derived adjunction:
                        $$
                            \begin{tikzcd}
                            	{\Dmod(X)} & {\U(\g)\bimod}
                            	\arrow[""{name=0, anchor=center, inner sep=0}, "{\bfGamma}"', shift right=2, from=1-1, to=1-2]
                            	\arrow[""{name=1, anchor=center, inner sep=0}, "{\D_X \tensor^{\L}_{\U(\g)} -}"', shift right=2, from=1-2, to=1-1]
                            	\arrow["\dashv"{anchor=center, rotate=-90}, draw=none, from=1, to=0]
                            \end{tikzcd}
                        $$   
                    thanks to the right/left-exactness (with respect to the canonical t-structures) of the composite functors (the composition of the two tensoring-up functors is trivially right-exact, and the composition $\R\Gamma(G, -) \circ \R\pi^*$ is left-exact since both factors are \textit{a priori} left-exact functors).  
                \end{proof}
            \begin{corollary}[An application to flag varieties]
                Since $G/B$ is a smooth variety, theorem \ref{theorem: localisation_adjunction_for_D_modules} specialises to the case of $X \cong G/B$.
            \end{corollary}
            
            It now remains to establish a functor from the category of $\U(\g)$-bimodules to the category $\calV$, which shall be $\Hom_{\U(\g)}(\V_{\lambda}, -)$ where $\V_{\lambda}$ is the Verma module attached to a weight $\lambda$. 
            \begin{theorem}[The Beilinson-Bernstein Localisation Theorem] \label{theorem: beilinson_bernstein_localisation}
                
            \end{theorem}
                \begin{proof}
                    
                \end{proof}
            \begin{corollary}[The Borel-Weil-Bott Theorem] \label{coro: borel_weil_bott_theorem}
                
            \end{corollary}